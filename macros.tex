\usepackage{ntheorem}
\RequirePackage{amsmath,amssymb,amsfonts,natbib}
\usepackage[a4paper,margin={.1\paperheight,.1\paperwidth},marginratio=1:1]{geometry}
\usepackage{enumerate}
%\usepackage{epsfig}

\usepackage{csquotes}
\usepackage{fontspec}
\usepackage{polyglossia}

%\setdefaultlanguage{french}
%\setotherlanguage[variant=british]{english}

\usepackage{hyperref}
\usepackage{xcolor}





\newtheorem{definition}{Définition}
\newtheorem{notation}{Notation}
{\theorembodyfont{\rmfamily}\newtheorem{example}{Exemple}}
{\theorembodyfont{\rmfamily\small}\newtheorem{exercise}{Exercice}}
\newtheorem{lemma}{Lemme}
\newtheorem{proposition}{Proposition}
\newtheorem{corollary}{Corollaire}
{\theorembodyfont{\rmfamily}\newtheorem{remark}{Remarque}}
\newtheorem{theorem}{Théorème}
{\theorembodyfont{\rmfamily}\newtheorem{warning}{Avertissement}}

% % % %
% Code SQL
% % % %

\renewcommand{\arraystretch}{1.5}
\newcommand{\sel}[2]{\sigma_{\tt #1}\left(\tt #2\right)}
\newcommand{\pro}[2]{\pi_{\tt #1}\left(\tt{\tt #2}\right)}
\newcommand{\join}[3]{\tt{#2}\bowtie_{\tt #1}\tt{#3}}
\newcommand{\NULL}{{\tt NULL}}
\newcommand{\dep}[2]{#1 \longrightarrow #2}


\usepackage{tikz}
\usetikzlibrary{shapes}

\newcommand{\field}[1]{\mathbb{#1}}
\newcommand{\R}{\field{R}}
\newcommand{\C}{\field{C}}
\newcommand{\Q}{\field{Q}}
\newcommand{\EE}{\field{E}}
\newcommand{\FF}{\field{F}}
\newcommand{\GG}{\field{G}}
\renewcommand{\L}{\field{L}}

\newcommand{\G}{{\mathcal G}}
\newcommand{\e}{{\mathcal E}}
\newcommand{\F}{{\mathcal F}}
\newcommand{\Loi}{{\mathcal L}}
\newcommand{\n}{{\mathcal N}}
\newcommand{\A}{{\mathcal A}}
\newcommand{\B}{{\mathcal B}}
\newcommand{\V}{{\mathcal V}}
\newcommand{\W}{{\mathcal V}}
\newcommand{\M}{{\mathcal M}}
\newcommand{\D}{\mathcal{D}}
\newcommand{\CC}{\mathcal{C}}
\newcommand{\X}{\mathcal{X}}

\newcommand{\set}[1]{\left\{#1\right\}}

\providecommand{\og}{\guillemotleft}
\providecommand{\fg}{\guillemotright}

% % % %
% % % %
\rhead{\textsf{Master I ISIFAR  \\2023--2024}}
\lhead{{\sf  ISIDATA8  Master I ISIFAR \\ TD
%\notd
}}
\cfoot{\thepage}


% % % %
\newcommand{\N}{\mathbb{N}}
\newcommand{\Z}{\mathbb{Z}}

\usepackage{textcomp}

\renewcommand{\contentsname}{}
